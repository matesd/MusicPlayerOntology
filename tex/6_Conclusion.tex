\chapter{Závěr}

co aplikace umi - vzit z readme.txt a poradne doplnit :)

+ jednoduche pristupne pouzitelne GUI

Nyni je ontologie naplnena nekolika testovacimi daty - po vetsim naplneni aplikace bude vracet jeste realnejsi vysledky a bude tak prinosnejsi pro bezne vyuziti.


kdyz jsme u toho naplnovani, lepsi nez rucni plneni dat by bylo lepsi odkazat napr. na DBpedii:

"

You can either define a genre by yourself, like this:

		:mygenre a mo:Genre; dc:title "electro rock".

		Or you can refer to a DBPedia genre (such as http://dbpedia.org/resource/Baroquemusic), allowing semantic web
		clients to access easily really detailed structured information about the genre you are refering to.
		
		"
		
Tento fakt jsem si uvedomil az v prubehu naplnovani dat - jelikoz mi slo predevsim o naplneni testovacich dat a otestovani vyhledavaci algoritmu, zustal jsem u rucniho naplnovani. Do budoucna prechod z rucniho plneni na odkazovani do napr. DBpedie/musicbrainz povazuji za jeden z nejdulezitejsich kroku!!!		

Do budoucna - [VIZ SESIT cast "future plan" !!!!!!!!!!!!!!!!!!!!!!!]

+ rozsirit ontologii (info o interpretovi, obrazky k interpretum, ..)

+ umoznit vytvaret instance (naplnovat data)

+ zkusili jsme si vyvinout vlastni hudebni ontologii - coz je super, naucili jsme se krasne jak to vsecko funguje - ale vysledek se hodne podoba musicontology ->

   rozhodně silněji propojit s musicontology - ne-li zcela nahradit (jaky je vyznam me ontologie, kdyz existuje uz jina dobra podobna) - zvazit, potreba diskuse. moznost celou moji ont. nahradit mo a web app ponechat (tzn. nebude vyhledavat v me ont, ale v mo). nebo lepe pouzit jako zaklad mo a rozsirit ji o me hudebni nastroje (pokud uz ale v mo taky neni nejak dobre udelano), napojeni na MO by take prineslo projektu podle me novy rozmer vyuziti (neco jineho je prijit s necim splacanym doma jako studentik, nez prijit "hele, mam tu system co vyhledava v svetove zname MO")

+ pro vyhledavani na dotaz "Doporuc mi nejakou hudbu - libi se mi Bobby McFerrin, mam rad Reggae, a nemam rad Pop" - ohodnoceni hran? viz kunc 3.1.3

      - pro tohle bych navrhoval nejspise plne vyuzit musicontology misto moji (mou bych musel dale vyvijet, udrzovat.., zatimco MO vyvijeji jini dobry lide, staraji se o ni, uz je hlavne vyvinuta a je mozne ji ihned nasadit) + doplnit mechanismy na tohle slozitejsi vyhledavani (pravd. ono ohodnoceni hran))
      
      Nepodarilo se mi zprovoznit Purl.org.
      
      
      v posledni rade provest refactoring kodu aplikace a presun na vlastni domenu (predevsim po prechodu z rucniho plneni dat na odkazovani do dbpedie/musicbrainz)
      
      
      
      + neco mas jeste na papire "osnova BP"
      
\begin{itemize}
\item Zhodnocení splnění cílů DP/BP a  vlastního přínosu práce (při formulaci je třeba vzít v potaz zadání práce).
\item Diskuse dalšího možného pokračování práce.
\end{itemize} 