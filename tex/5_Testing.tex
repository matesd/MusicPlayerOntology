\chapter{Testování}

[otestovat vypis songu]

inspirace u Petra Pokorneho, Tomika..

1) otestovat ontologii - W3C validator; algoritmus testnu tak, že naplnim hodně dat, vysledky se jednoduse daji porovnat pouhou znalosti interpretu a rict si "davaji vysledky smysl? jsou realne podle toho co vim?" - toto je nejjednodussi test. 
Lepsi test poskytne vytvoreni několik duplicitních interpretu - výsledek by měl zobrazit duplicity se 100\% shodou,
+ internet.php radek 150 /* Remove the searched interpret from final array */ zakomentovat a musi se na 1. miste zobrazit ten sami interpret. 

2) kognitivni pruchod (- to dost souvisi s bodem 1. takze to nak vymyslet dohromady?)
3) user testing (zda neco vyhledaji)
4) accessibility/usability (+strucne pojmy vysvetlit) - tzn. bez JS/CSS/WAI bo jak se to jmenuje proste to AA/AAA, barvicky, hendikepy apod. (pomoci pluginu do firefoxu?)

dal sem zjistil, ze uzivatelum je jedno zda jsou podobni na 33 ci 39proc, lepsi je viditelny ukazatel.

Dal testovani objevilo nespravnou funkci - odkaz z alba ci songu vede na stranku vyhledavani te same pisnicky/alba - to je blbost, uzivatel ocekava ze odkaz povede na info o albu - opraveno!

napr "rock \& roll",r \& b vs. "rock\&roll", r\&b - aplikace umi.

+ odlehcena vyhledavaci hlavicka (viz obrazky)


\begin{itemize}
 \item Způsob, průběh a výsledky testování.
 \item Srovnání s existujícími řešeními, pokud jsou známy.
\end{itemize} 

