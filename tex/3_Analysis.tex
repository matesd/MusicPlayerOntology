\chapter{Analýza a návrh řešení}

\section{Představa aplikace}

Navržená ontologie hudby by pro naše účely hudebního přehrávače měla popisovat hudbu takovým způsobem, aby pokryla požadavky a očekávání uživatele hudebního přehrávače.


Potenciální posluchač, než začne samotnou hudbu poslouchat, potřebuje napřed požadovanou hudbu nalézt (ať již na svém kapesním přehrávači či ve webové aplikaci).
Nejjednodušší cesta k nalezení konkrétní hudby samozřejmě nastává v situaci, kdy uživatel přesně ví, jaký název písně či interpreta požaduje. Uživatel si ale pouze s tímto základním vyhledáváním nevystačí.
Uživatel může mít např. chuť poslouchat jeho oblíbený žánr a nezáleží mu na konkrétním interpretovi. 

Nebo naopak. V běžném životě si člověk zcela přirozeně řekne, že se mu líbí určitý hudební interpret, a že by rád poslouchal jemu podobnou hudbu. Proto stejně tak přirozené by mělo být tuto myšlenku převést do hudebního přehrávače.
Převeďme si tuto myšlenku do věty v přirozeném jazyce, tedy např. "Líbí se mi hudba Michaela Jacksona". 
Pokud se vám někdo tuto větu řekne a zároveň vás požádá, abyste mu sdělili podobnou hudbu, jste schopni mu odpovědět, a to na základě znalostí, které máte, či které si dohledáte.
 
Stejnou komunikaci mezi lidmi se tedy pokusíme převést do počítačové podoby. 
Dotázaný člověk v našem příkladu, jak již bylo řečeno, odpovídal na základě \textit{znalostí}. Bez potřebných znalostí by totiž nemohl tazateli odpovědět.
Pro úspěšné převedení takové komunikace tedy evidentně musíme počítači zajistit potřebné znalosti, aby i on věděl stejně jako vědí lidé, že Michaelu Jacksonovi je podobná např. Whitney Houston s Madonnou, zatímco např. takový Marilyn Manson s ním nemá nic společného. 

Ideálním nástrojem pro počítač, aby mohl odvozovat takovéto vzájemné vztahy, je zcela jistě ontologie.
Pomocí ontologie totiž můžeme vyrobit potřebnou znalostní bázi, z které pak náš program bude čerpat a na jejím základě může vyhodnocovat potřebné znalosti a ve výsledku tak nabízet pro uživatele relevantní informace.

Uživatel může mít dále touhu poslouchat směsici různých žánrů. Má chuť např. na britský pop říznutý Rock&Rollem.
Či je uživatel dokonce milovník a znalec konkrétních hudebních nástrojů

\section{Současná řešení}

    \subsection{FreeDB,freebase ?}
    
    \subsection{MusicBrainz}
    
    \subsection{Music Ontology}

    \subsection{BBC music, last.fm, [podivat se do bookmarku]}


\section{FOAF..}


\section{EVENT..}

\section{Návrh hudební ontologie}


vývoj, diagramy, ..
inspirace v musicontology


Výsledek do Ontology Browser (CO-ODE Projekt)a do OWLDoc.

\section{Návrh webové aplikace}